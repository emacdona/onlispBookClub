% No document class! This file is intended for inclusion in one of the other tex
% files in this directory

\usepackage[utf8]{inputenc}
\usepackage{minted}
\usepackage{listings}
\usepackage{graphicx}
\usepackage{xcolor}
\usepackage{adjustbox}

\usetheme{Madrid}
\useinnertheme{circles}
\definecolor{ssgreen}{HTML}{669B41}
\usecolortheme[named=ssgreen]{structure}

%Change link colors, except for navigation links...
\definecolor{links}{HTML}{2A1B81}
\hypersetup{colorlinks,linkcolor=,urlcolor=links}
%And, except for footer links
\addtobeamertemplate{footline}{\hypersetup{allcolors=.}}{}

\setbeamertemplate{navigation symbols}{}
\setlength{\columnseprule}{0.4pt}

\AtBeginEnvironment{frame}{\setcounter{footnote}{0}}

\title[Sandbox]{My Docker and Kubernetes Sandbox}
\author[Ed MacDonald]{Ed MacDonald\\emacdonald@solutionstreet.com}
\institute[\href{https://solutionstreet.com}{SolutionStreet}]{SolutionStreet\\\href{https://solutionstreet.com}{(solutionstreet.com)}}
\date{December 2022}

\titlegraphic{ \includegraphics[width=2cm]{logo} }

% Notes:
% sync user in container w/ user in host
% docker beside docker
% sync fs locations in/out of container so other ``beside'' containers see same files
% XServer
% Quick overview of docker..

\begin{document}
    \frame{\titlepage}

    \begin{frame}
      \frametitle{Why Do You Need a Sandbox?}
      \begin{itemize}
      \item<2->How many versions of Python/Java/Perl/Ruby/Node are on your
        machine right now? Where are they installed?
      \item<3->If you got a new machine today, how long would it take until all
        of your existing projects built on it?
      \item<4->Is there any software on your machine that you downloaded once and forgot about?
      \item<5->Do you love experimenting with new tech?
      \end{itemize}
      \note<5->[item]{FYI: This entire presentation was built solely with tools running
        in the sandbox.}
    \end{frame}

    \begin{frame}
    \frametitle{What is Linux?}
    \begin{columns}
        \begin{column}{0.5\textwidth}
            \begin{itemize}
                \item<2->Is Redhat Linux?
                \item<3->Is Ubuntu Linux?
                \item<4->Is Debian Linux?
                \item<5->Is Alpine Linux?
            \end{itemize}
        \end{column}
        \begin{column}{0.5\textwidth}
          \begin{center}
            \only<6->{\Huge \color{red} NO!!}
          \end{center}
        \end{column}
    \end{columns}
    \end{frame}

    \begin{frame}
    \frametitle{What is Linux?}
    \begin{itemize}
        \item<1-> Linux is a kernel
        \item<2-> A kernel is a set of system calls providing a hardware abstraction
        \item<3-> Great efforts are made to keep system calls backwards compatible
    \end{itemize}
    \note<3->[item]{Linux maintainers FEEL VERY STRONGLY that new kernel versions should strive to maintain binary compatibility}
    \end{frame}

    \begin{frame}
      \frametitle{Why Does That Matter?}
      \begin{itemize}
      \item<1-> Docker images don't have a kernel
      \item<2-> They have a filesystem with binaries
      \item<3-> You can reasonably expect a Docker image based on an old kernel to run on newer one
      \end{itemize}
      \note<1->[item]{one}
      \note<2->[item]{two}
      \note<3->[item]{three}
    \end{frame}

    % https://tex.stackexchange.com/questions/156480/updating-picture-after-pause
    \begin{frame}
      \frametitle{Baseline}
      \begin{columns}
        \begin{column}{0.5\textwidth}
          \includegraphics<1| handout:0>[width=\textwidth,height=0.85\textheight,keepaspectratio]{../graphics/010.eps}
          \includegraphics<2| handout:0>[width=\textwidth,height=0.85\textheight,keepaspectratio]{../graphics/020.eps}
          \includegraphics<3| handout:0>[width=\textwidth,height=0.85\textheight,keepaspectratio]{../graphics/030.eps}
          \includegraphics<4| handout:0>[width=\textwidth,height=0.85\textheight,keepaspectratio]{../graphics/040.eps}
          \includegraphics<5| handout:0>[width=\textwidth,height=0.85\textheight,keepaspectratio]{../graphics/050.eps}
          \includegraphics<6>[width=\textwidth,height=0.85\textheight,keepaspectratio]{../graphics/060.eps}
        \end{column}
        \begin{column}{0.5\textwidth}
          \begin{itemize}
          \item<1-> Windows
          \item<2-> XServer
          \item<3-> WSL
          \item<4-> User and Home Directory
          \item<5-> Docker CLI
          \item<6-> Docker
          \end{itemize}
        \end{column}
      \end{columns}
      \note<1->[item]{Whenever I say ``Host'' (unqualified), I'm referring to WSL}
      \note<4->[item]{WSL user, but I'm guessing it maps to a Windows user}
      \note<4->[item]{WSL directory, but I'm guessing it maps to a Windows directory}
    \end{frame}

    \begin{frame}
      \frametitle{A Word on Operating Systems}
      \begin{itemize}
      \item<2-> Linux: clearly the best choice for this
      \item<3-> Windows: surprisingly great
        \begin{itemize}
        \item<4-> WSL is \textit{very} nice
        \item<5-> WSL Appears to share all resources (CPU; Memory) with Windows
        \item<6-> VcXsrv is \textit{very} capable
        \end{itemize}
      \item<7-> OS X: surprisingly bad
        \begin{itemize}
        \item<8-> Docker desktop ``takes'' resources from host
        \item<9-> XQuartz was not as nice as VcXsrv --- at least in Ventura
        \end{itemize}
      \end{itemize}
      \note<3->[item]{Why did I choose Windows initially? Because like most gamers, my Windows PC is the most powerful machine I have.}
    \end{frame}

    \begin{frame}
      \frametitle{Sandbox}
      \begin{columns}
        \begin{column}{0.5\textwidth}
          \includegraphics<1| handout:0>[width=\textwidth,height=0.85\textheight,keepaspectratio]{../graphics/070.eps}
          \includegraphics<2| handout:0>[width=\textwidth,height=0.85\textheight,keepaspectratio]{../graphics/080.eps}
          \includegraphics<3| handout:0>[width=\textwidth,height=0.85\textheight,keepaspectratio]{../graphics/090.eps}
          \includegraphics<4| handout:0>[width=\textwidth,height=0.85\textheight,keepaspectratio]{../graphics/100.eps}
          \includegraphics<5>[width=\textwidth,height=0.85\textheight,keepaspectratio]{../graphics/110.eps}
        \end{column}
        \begin{column}{0.5\textwidth}
          \begin{itemize}
          \item<1-> Container
          \item<2-> Programs
          \item<3-> User and Home Directory ...
          \item<4-> ... Mapped to those on the host
          \item<5-> Docker CLI talks to host's dockerd
          \end{itemize}
        \end{column}
        \note<2->[item]{Of particular interest:\\Docker CLI\\supervisord}
      \end{columns}
    \end{frame}

    \begin{frame}
      \frametitle{Docker-beside-Docker\texttrademark}
      \begin{itemize}
      \item If, from inside a container, you want to be able to:\pause
        \begin{itemize}
        \item Create new containers\pause
        \item Share files in your container with these new containers\pause
        \end{itemize}
        \item You don't need Docker-in-Docker
      \end{itemize}
    \end{frame}

    \begin{frame}
      \frametitle{Docker-beside-Docker\texttrademark}
      \begin{itemize}
      \item<1->Install the docker CLI in your container
      \item<2->Mount the host's docker unix domain socket into the container
      \item<3->Ensure directories in the container have the same name as directories on the host
      \end{itemize}
      \note<3->[item]{If you follow these instructions, containers you create with
        the docker cli in your container --- with bind mounts to directories in
        your container --- will actually be created by the same dockerd your
        container was created with. Their bind mounts will technically see files
        on the host, but those will be the same files -- with the same names -- as
        those in your container.}
    \end{frame}

    \begin{frame}
      \frametitle{Demo I}
      \begin{center}
        {\Huge Demo I}
      \end{center}
      \note[item]{supervisord/supervisorctl}
      \note[item]{docker-beside-docker: docker run -it -v `pwd`:/workdir -w /workdir ubuntu ls}
      \note[item]{'docker ps' from host and container}
      \note[item]{'lazydocker'; show DISPLAY env var; show 'top'}
      \note[item]{sc status}
      \note[item]{sc start emacs}
      \note[item]{sc start dia}
      \note[item]{sc start firefox --- Will be able to enter Intellij auth; Useful for Demo II}
      \note[item]{sc start intellij --- will be able to show ansible config in Demo II}
      \note[item]{sc start qtcreator}
    \end{frame}

    \begin{frame}
      \frametitle{Kubernetes in the Sandbox}
      \begin{columns}
        \begin{column}{0.5\textwidth}
          \includegraphics<1| handout:0>[width=\textwidth,height=0.85\textheight,keepaspectratio]{../graphics/120.eps}
          \includegraphics<2>[width=\textwidth,height=0.85\textheight,keepaspectratio]{../graphics/130.eps}
        \end{column}
        \begin{column}{0.5\textwidth}
          \begin{itemize}
          \item<1-> Another docker container...
          \item<2-> ...Running K3d
          \end{itemize}
        \end{column}
      \end{columns}
    \end{frame}

    \begin{frame}
      \frametitle{Demo II}
      \begin{center}
        {\Huge Demo II}
      \end{center}
      \note[item]{provision k3d cluster}
      \note[item]{watch lazydocker}
      \note[item]{watch k9s}
      \note[item]{show deployed apps --- if they come up in time}
      \note[item]{deploy lisp ``hello world'' app and show result}
    \end{frame}

    \begin{frame}
      \frametitle{Resources and Tools Used}
      \begin{itemize}
      \item lazydocker:
        \href{https://github.com/jesseduffield/lazydocker}{https://github.com/jesseduffield/lazydocker}
      \item k3d:
        \href{https://k3d.io/}{https://k3d.io/}
      \item \LaTeX:
        \href{https://www.latex-project.org/}{https://www.latex-project.org/}
      \item Beamer:
        \href{https://ctan.org/pkg/beamer?lang=en}{https://ctan.org/pkg/beamer?lang=en}
      \item Pympress:
        \href{https://github.com/Cimbali/pympress}{https://github.com/Cimbali/pympress}
      \item 12-bit Rainbow Palette:
        \href{https://iamkate.com/data/12-bit-rainbow/}{https://iamkate.com/data/12-bit-rainbow/}
      \end{itemize}
    \end{frame}

    \begin{frame}
      \frametitle{Questions}
      \begin{center}
        {\Huge Questions ?}
      \end{center}
    \end{frame}

\end{document}
