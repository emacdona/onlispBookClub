% No document class! This file is intended for inclusion in one of the other tex
% files in this directory

\usepackage[utf8]{inputenc}
\usepackage{minted}
\usepackage{listings}
\usepackage{graphicx}
\usepackage{xcolor}
\usepackage{adjustbox}

\usetheme{Madrid}
\useinnertheme{circles}
\definecolor{ssgreen}{HTML}{669B41}
\usecolortheme[named=ssgreen]{structure}

%Change link colors, except for navigation links...
\definecolor{links}{HTML}{2A1B81}
\hypersetup{colorlinks,linkcolor=,urlcolor=links}
%And, except for footer links
\addtobeamertemplate{footline}{\hypersetup{allcolors=.}}{}

\setbeamertemplate{navigation symbols}{}
\setlength{\columnseprule}{0.4pt}

\AtBeginEnvironment{frame}{\setcounter{footnote}{0}}

\title[Sandbox]{My Docker and Kubernetes Sandbox}
\author[Ed MacDonald]{Ed MacDonald\\emacdonald@solutionstreet.com}
\institute[\href{https://solutionstreet.com}{SolutionStreet}]{SolutionStreet\\\href{https://solutionstreet.com}{(solutionstreet.com)}}
\date{December 2022}

%\titlegraphic{ \includegraphics[width=2cm]{logo} }

% Notes:
% sync user in container w/ user in host
% docker beside docker
% sync fs locations in/out of container so other ``beside'' containers see same files
% XServer
% Quick overview of docker..

\begin{document}
    \frame{\titlepage}

    \begin{frame}
      \frametitle{Why Do You Need a Sandbox?}
      \begin{itemize}
      \pause
      \item How many versions of Python/Java/Perl/Ruby/Node are on your machine right now?\pause
      \item If you got a new machine today, how long would it take until all your existing projects built on it?\pause
      \item Is there any software on your machine that you downloaded once and forgot about?\pause
      \item Do you love experimenting with new tech?
      \end{itemize}
      \note[item]{Hello World}
    \end{frame}

    \begin{frame}
    \frametitle{What is Linux?}
    \begin{columns}
        \begin{column}{0.5\textwidth}
            \begin{itemize}
                \item Is Redhat Linux?\pause
                \item Is Ubuntu Linux?\pause
                \item Is Debian Linux?\pause
                \item Is Alpine Linux?\pause
            \end{itemize}
        \end{column}
        \begin{column}{0.5\textwidth}
          \begin{center}
            {\Huge \color{red} NO!!}
          \end{center}
        \end{column}
    \end{columns}
    \note[item]{Hello World}
    \end{frame}

    \begin{frame}
    \frametitle{What is Linux?}
    \begin{itemize}
        \item Linux is a kernel\pause
        \item A kernel is a set of system calls providing a hardware abstraction\pause
        \item Linux maintainers FEEL VERY STRONGLY that new kernel versions should strive to maintain binary compatibility
    \end{itemize}
    \end{frame}

    \begin{frame}
      \frametitle{Why Does That Matter?}
      \begin{itemize}
      \item Docker images don't have a kernel\pause
      \item They have a filesystem with binaries\pause
      \item You can reasonably expect a Docker image based on an old kernel to run on newer one
      \end{itemize}
    \end{frame}

    % https://tex.stackexchange.com/questions/156480/updating-picture-after-pause
    \begin{frame}
      \frametitle{Baseline}
      \begin{columns}
        \begin{column}{0.5\textwidth}
          \includegraphics<1| handout:0>[width=\textwidth,height=0.85\textheight,keepaspectratio]{../graphics/010.eps}
          \includegraphics<2| handout:0>[width=\textwidth,height=0.85\textheight,keepaspectratio]{../graphics/020.eps}
          \includegraphics<3| handout:0>[width=\textwidth,height=0.85\textheight,keepaspectratio]{../graphics/030.eps}
          \includegraphics<4| handout:0>[width=\textwidth,height=0.85\textheight,keepaspectratio]{../graphics/040.eps}
          \includegraphics<5| handout:0>[width=\textwidth,height=0.85\textheight,keepaspectratio]{../graphics/050.eps}
          \includegraphics<6>[width=\textwidth,height=0.85\textheight,keepaspectratio]{../graphics/060.eps}
        \end{column}
        \begin{column}{0.5\textwidth}
          \begin{itemize}
          \item <1-> Windows
          \item <2-> XServer
          \item <3-> WSL
          \item <4-> A user
          \item <4-> That user's home directory
          \item <5-> Docker CLI
          \item <6-> Docker
          \end{itemize}
        \end{column}
      \end{columns}
      \note[item]{Whenever I say ``Host'' (unqualified), I'm referring to WSL}
      \note[item]{WSL user, but I'm guessing it maps to a Windows user}
      \note[item]{WSL directory, but I'm guessing it maps to a Windows directory}
    \end{frame}

    \begin{frame}
      \frametitle{A Word on Operating Systems}
      \begin{itemize}\pause
      \item Linux: clearly the best choice for this\pause
      \item Windows: surprisingly great\pause
        \begin{itemize}
        \item WSL is \textit{very} nice\pause
        \item WSL Appears to share all resources (CPU; Memory) with Windows\pause
        \item VcXsrv is \textit{very} capable\pause
        \end{itemize}
      \item OS X: surprisingly bad\pause
        \begin{itemize}
        \item Docker desktop ``takes'' resources from host\pause
        \item XQuartz was not as nice as VcXsrv --- at least in Ventura
        \end{itemize}
      \end{itemize}
      \note[item]{Why did I choose Windows initially? Because like most gamers, my Windows PC is the most powerful machine I have.}
    \end{frame}

    \begin{frame}
      \frametitle{Sandbox}
      \begin{columns}
        \begin{column}{0.5\textwidth}
          \includegraphics<1| handout:0>[width=\textwidth,height=0.85\textheight,keepaspectratio]{../graphics/070.eps}
          \includegraphics<2| handout:0>[width=\textwidth,height=0.85\textheight,keepaspectratio]{../graphics/080.eps}
          \includegraphics<3| handout:0>[width=\textwidth,height=0.85\textheight,keepaspectratio]{../graphics/090.eps}
          \includegraphics<4| handout:0>[width=\textwidth,height=0.85\textheight,keepaspectratio]{../graphics/100.eps}
          \includegraphics<5>[width=\textwidth,height=0.85\textheight,keepaspectratio]{../graphics/110.eps}
        \end{column}
        \begin{column}{0.5\textwidth}
          \begin{itemize}
          \item <1-> Container
          \item <2-> Programs
          \item <3-> User and Home Directory ...
          \item <4-> ... Mapped to those on the host
          \item <5-> Docker CLI talks to host's dockerd
          \end{itemize}
        \end{column}
      \end{columns}
      \note[item]{Whenever I say ``Host'' (unqualified), I'm referring to WSL}
      \note[item]{WSL user, but I'm guessing it maps to a Windows user}
      \note[item]{WSL directory, but I'm guessing it maps to a Windows directory}
    \end{frame}

    \begin{frame}
      \frametitle{Kubernetes in the Sandbox}
      \begin{columns}
        \begin{column}{0.5\textwidth}
          \includegraphics<1| handout:0>[width=\textwidth,height=0.85\textheight,keepaspectratio]{../graphics/120.eps}
          \includegraphics<2>[width=\textwidth,height=0.85\textheight,keepaspectratio]{../graphics/130.eps}
        \end{column}
        \begin{column}{0.5\textwidth}
          \begin{itemize}
          \item <1-> Another docker container...
          \item <2-> ...Running K3d
          \end{itemize}
        \end{column}
      \end{columns}
      \note[item]{Whenever I say ``Host'' (unqualified), I'm referring to WSL}
      \note[item]{WSL user, but I'm guessing it maps to a Windows user}
      \note[item]{WSL directory, but I'm guessing it maps to a Windows directory}
    \end{frame}

\end{document}
