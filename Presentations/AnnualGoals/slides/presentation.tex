% No document class! This file is intended for inclusion in one of the other tex
% files in this directory

\usepackage[utf8]{inputenc}
\usepackage{minted}
\usepackage{listings}
\usepackage{graphicx}
\usepackage{xcolor}
\usepackage{adjustbox}

\usetheme{Madrid}
\useinnertheme{circles}
\definecolor{ssgreen}{HTML}{669B41}
\usecolortheme[named=ssgreen]{structure}

%Change link colors, except for navigation links...
\definecolor{links}{HTML}{2A1B81}
\hypersetup{colorlinks,linkcolor=,urlcolor=links}
%And, except for footer links
\addtobeamertemplate{footline}{\hypersetup{allcolors=.}}{}

\setbeamertemplate{navigation symbols}{}
\setlength{\columnseprule}{0.4pt}

\AtBeginEnvironment{frame}{\setcounter{footnote}{0}}

\title[Sandbox]{My Docker and Kubernetes Sandbox}
\author[Ed MacDonald]{Ed MacDonald\\emacdonald@solutionstreet.com}
\institute[\href{https://solutionstreet.com}{SolutionStreet}]{SolutionStreet\\\href{https://solutionstreet.com}{(solutionstreet.com)}}
\date{December 2022}

%\titlegraphic{ \includegraphics[width=2cm]{logo} }

% Notes:
% sync user in container w/ user in host
% docker beside docker
% sync fs locations in/out of container so other ``beside'' containers see same files
% XServer
% Quick overview of docker..

\begin{document}
    \frame{\titlepage}

    \begin{frame}
      \frametitle{Why Do You Need a Sandbox?}
      \begin{itemize}
      \pause
      \item How many versions of Python/Java/Perl/Ruby/Node are on your machine right now?\pause
      \item If you got a new machine today, how long would it take until all your existing projects built on it?\pause
      \item Is there any software on your machine that you downloaded once and forgot about?\pause
      \item Do you love experimenting with new tech?
      \end{itemize}
      \note[item]{Hello World}
    \end{frame}

    \begin{frame}
    \frametitle{What is Linux?}
    \begin{columns}
        \begin{column}{0.5\textwidth}
            \begin{itemize}
                \item Is Redhat Linux?\pause
                \item Is Ubuntu Linux?\pause
                \item Is Debian Linux?\pause
                \item Is Alpine Linux?\pause
            \end{itemize}
        \end{column}
        \begin{column}{0.5\textwidth}
          \begin{center}
            {\Huge \color{red} NO!!}
          \end{center}
        \end{column}
    \end{columns}
    \note[item]{Hello World}
    \end{frame}

    \begin{frame}
    \frametitle{What is Linux?}
    \begin{itemize}
        \item Linux is a kernel\pause
        \item A kernel is a set of system calls providing a hardware abstraction\pause
        \item Linux maintainers FEEL VERY STRONGLY that new kernel versions should strive to maintain binary compatibility
    \end{itemize}
    \end{frame}

    \begin{frame}
      \frametitle{Why Does That Matter?}
      \begin{itemize}
      \item Docker images don't have a kernel\pause
      \item They have a filesystem with binaries\pause
      \item You can reasonably expect a Docker image based on an old kernel to run on newer one
      \end{itemize}
    \end{frame}

    \begin{frame}
      \frametitle{Baseline}
      \begin{columns}
        \begin{column}{0.5\textwidth}
          \includegraphics[width=\textwidth,height=0.85\textheight,keepaspectratio]{../graphics/windows-wsl.eps}
        \end{column}
        \begin{column}{0.5\textwidth}
          \begin{itemize}
          \item Windows\pause
          \item WSL\pause
          \item A user\pause
          \item That user's home directory\pause
          \item docker CLI
          \end{itemize}
        \end{column}
      \end{columns}
      \note[item]{Whenever I say ``Host'' (unqualified), I'm referring to WSL}
      \note[item]{WSL user, but I'm guessing it maps to a Windows user}
      \note[item]{WSL directory, but I'm guessing it maps to a Windows directory}
    \end{frame}

    \begin{frame}
      \frametitle{Docker}
      \begin{columns}
        \begin{column}{0.5\textwidth}
          \includegraphics[width=\textwidth,height=0.85\textheight,keepaspectratio]{../graphics/windows-wsl-docker.eps}
        \end{column}
        \begin{column}{0.5\textwidth}
          \begin{itemize}
          \item dockerd runs on the \textbf{host}
          \end{itemize}
        \end{column}
      \end{columns}
      \note[item]{I point out that it runs on the host to draw the distinction from ``docker in docker''}
    \end{frame}

    \begin{frame}
      \frametitle{Applications}
      \begin{columns}
        \begin{column}{0.5\textwidth}
          \includegraphics[width=\textwidth,height=0.85\textheight,keepaspectratio]{../graphics/windows-wsl-docker-container-A.eps}
        \end{column}
        \begin{column}{0.5\textwidth}
          \begin{itemize}
          \item Item 1
          \item Item 2
          \item Item 3
          \item Item 4
          \end{itemize}
        \end{column}
      \end{columns}
    \end{frame}

    \begin{frame}
      \frametitle{``Docker Beside Docker''}
      \begin{columns}
        \begin{column}{0.5\textwidth}
          \includegraphics[width=\textwidth,height=0.85\textheight,keepaspectratio]{../graphics/windows-wsl-docker-container-A-user.eps}
        \end{column}
        \begin{column}{0.5\textwidth}
          \begin{itemize}
          \item Item 1
          \item Item 2
          \item Item 3
          \item Item 4
          \end{itemize}
        \end{column}
      \end{columns}
    \end{frame}

    \begin{frame}
      \frametitle{Title!}
      \begin{columns}
        \begin{column}{0.5\textwidth}
          \includegraphics[width=\textwidth,height=0.85\textheight,keepaspectratio]{../graphics/windows-wsl-docker-container-A-user-windowsx.eps}
        \end{column}
        \begin{column}{0.5\textwidth}
          \begin{itemize}
          \item Item 1
          \item Item 2
          \item Item 3
          \item Item 4
          \end{itemize}
        \end{column}
      \end{columns}
    \end{frame}

    \begin{frame}
      \frametitle{Title!}
      \begin{columns}
        \begin{column}{0.5\textwidth}
          \includegraphics[width=\textwidth,height=0.85\textheight,keepaspectratio]{../graphics/windows-wsl-docker-container-A-user-windowsx-k3d.eps}
        \end{column}
        \begin{column}{0.5\textwidth}
          \begin{itemize}
          \item Item 1
          \item Item 2
          \item Item 3
          \item Item 4
          \end{itemize}
        \end{column}
      \end{columns}
    \end{frame}

\end{document}
