% No document class! This file is intended for inclusion in one of the other tex
% files in this directory

\usepackage[utf8]{inputenc}
\usepackage{minted}
\usepackage{listings}
\usepackage{graphicx}
\usepackage{xcolor}
\usepackage{adjustbox}

% https://tex.stackexchange.com/questions/2369/why-do-the-less-than-symbol-and-the-greater-than-symbol-appear-wrong-as
\usepackage[T1]{fontenc}

\usetheme{Madrid}
\useinnertheme{circles}
\definecolor{ssgreen}{HTML}{669B41}
\usecolortheme[named=ssgreen]{structure}

%Change link colors, except for navigation links...
\definecolor{links}{HTML}{2A1B81}
\hypersetup{colorlinks,linkcolor=,urlcolor=links}
%And, except for footer links
\addtobeamertemplate{footline}{\hypersetup{allcolors=.}}{}

\setbeamertemplate{navigation symbols}{}
\setlength{\columnseprule}{0.4pt}

\AtBeginEnvironment{frame}{\setcounter{footnote}{0}}

\title[Sandbox]{My Docker and Kubernetes Sandbox}
\author[Ed MacDonald]{Ed MacDonald\\emacdonald@solutionstreet.com}
\institute[\href{https://solutionstreet.com}{SolutionStreet}]{SolutionStreet\\\href{https://solutionstreet.com}{(solutionstreet.com)}}
\date{December 2022}

\titlegraphic{ \includegraphics[width=2cm]{logo} }

% Frames vs Slides; using slides to expose ever more developed graphics;
% choosing which slide is used to render handout...
% https://tex.stackexchange.com/questions/156480/updating-picture-after-pause
% https://www.overleaf.com/learn/latex/Beamer
% https://tex.stackexchange.com/questions/120866/how-to-handle-only-overlays-while-printing-a-latex-beamer-notes

% Beamer slide overlays
% https://www.texdev.net/2014/01/17/the-beamer-slide-overlay-concept/
% https://www.overleaf.com/learn/latex/Beamer_Presentations%3A_A_Tutorial_for_Beginners_(Part_4)%E2%80%94Overlay_Specifications

\begin{document}
    \frame{\titlepage}

    \begin{frame}
      \frametitle{Why Do You Need a Sandbox?}
      \begin{itemize}
      \item<2->How many versions of Python/Java/Perl/Ruby/Node are on your
        machine right now? Where are they installed?
      \item<3->If you got a new machine today, how long would it take until all
        of your existing projects built on it?
      \item<4->Is there any software on your machine that you downloaded once and forgot about?
      \item<5->Do you love experimenting with new tech?
      \end{itemize}
      \note<5->[item]{FYI: This entire presentation was built solely with tools running
        in the sandbox.}
    \end{frame}

    % https://tex.stackexchange.com/questions/384722/code-in-beamer-presentation
    \begin{frame}[fragile]
      \frametitle{What Drove Me to Make This Sandbox?}
      \begin{itemize}
      \item<2->
        {\color<3-| handout:0>{gray!30}
          When I mess around with new software, it's not always clear to me that
          my first attempt at installing and configuring it will be my ``best''
          attempt.
        }
      \item<3->
        {\color<4-| handout:0>{gray!30}
          This sandbox started as a Common Lisp development environment
          (SBCL\footnote<3->{Steel Bank Common Lisp: \href{https://www.sbcl.org/}{https://www.sbcl.org/}}, Spacemacs) --- something which I knew very little about. I knew
          it would take a few iterations to configure it how I wanted.
        }
      \item<4->
        {\color<5-| handout:0>{gray!30}
          It quickly evolved into a Kubernetes playground --- something
          even more daunting to configure.
        }
      \item<5->
        {\color<6-| handout:0>{gray!30}
          If I know I can clean something up easily, I'm more inclined to
          install it and try it out.
        }
      \item<6->Cleaning up this sandbox is as easy as:\\
\begin{minted}{bash}
docker kill <container> && docker rmi <image>
\end{minted}
      \end{itemize}
      \note<5->[item]{I installed minikube once, and I wasn't sure how to remove
      all traces of it from my machine.}
    \end{frame}

    \begin{frame}
      \frametitle{Demo I}
      \begin{center}
        {\Huge Demo I}\\
        This presentation is running in the sandbox --- so if you can read this, the
        demo is actually already underway.
      \end{center}
      \note[item]{run 'lazydocker'\\show 'top'\\show 'pympress' in list}
      \note[item]{sc status}
      \note[item]{sc start dia}
    \end{frame}

    \begin{frame}
    \frametitle{What is Linux?}
    \begin{columns}
        \begin{column}{0.5\textwidth}
            \begin{itemize}
                \item<2->Is Redhat Linux?
                \item<3->Is Ubuntu Linux?
                \item<4->Is Debian Linux?
                \item<5->Is Alpine Linux?
            \end{itemize}
        \end{column}
        \begin{column}{0.5\textwidth}
          \begin{center}
            \only<6->{\Huge \color{red} NO!!}
          \end{center}
        \end{column}
    \end{columns}
    \end{frame}

    \begin{frame}
    \frametitle{What is Linux?}
    \begin{itemize}
        \item<1-> Linux is a kernel.
        \item<2-> A kernel is a set of system calls providing a hardware abstraction.
        \item<3-> Great efforts are made to keep Linux system calls backwards compatible.
    \end{itemize}
    \note<3->[item]{Linux maintainers FEEL VERY STRONGLY that new kernel versions should strive to maintain binary compatibility.}
    \end{frame}

    \begin{frame}
      \frametitle{Why Does That Matter?}
      \begin{itemize}
      \item<1-> Docker images don't have a kernel.
      \item<2-> They have a filesystem with binaries.
      \item<3-> You can reasonably expect a Docker image based on an old kernel to run on newer one.
      \item<4-> This makes it a great platform for a Sandbox:
        \begin{itemize}
          \item<5-> Lump all your software and its configuration in a container.
          \item<6-> You can reasonably expect that container to run on future
            kernel versions.
          \end{itemize}
      \end{itemize}
    \end{frame}

    % https://tex.stackexchange.com/questions/156480/updating-picture-after-pause
    \begin{frame}
      \frametitle{Baseline}
      \begin{columns}[T]
        \begin{column}{0.5\textwidth}
          \begin{overprint}
          \includegraphics<1| handout:0>[width=\textwidth,height=0.85\textheight,keepaspectratio]{../graphics/010.eps}
          \includegraphics<2->[width=\textwidth,height=0.85\textheight,keepaspectratio]{../graphics/020.eps}
          \end{overprint}
        \end{column}
        \begin{column}{0.5\textwidth}
          \begin{overprint}
          \begin{itemize}
          \item<1-> Windows
          \item<2-> X Server
          \end{itemize}
          \end{overprint}
        \end{column}
      \end{columns}
    \end{frame}

    \begin{frame}
      \frametitle{Hold On. X-what?}
      \begin{itemize}
      \item<2->
        {\color<3-| handout:0>{gray!30}
          X Server.}
      \item<3->
        {\color<4-| handout:0>{gray!30}
          Once upon a time, kernels and windowing systems were developed
          completely independently of one-another.
        }
      \item<4->
        {\color<5-| handout:0>{gray!30}
          Programs that need to draw windows on a display (X Clients)
          just send instructions to an X Server that knows how to do so.
        }
      \item<5->
        {\color<6-| handout:0>{gray!30}
          The Server sends things like mouse events back to the clients.
        }
      \item<6->
        {\color<7-| handout:0>{gray!30}
          WSL\footnote<6->{Windows Subsystem for Linux: \href{https://learn.microsoft.com/en-us/windows/wsl/install}{https://learn.microsoft.com/en-us/windows/wsl/install}} doesn't have access to the display, that's Windows'
          responsibility.
        }
      \item<7->So we install an X Server that runs on Windows and have WSL (and
        Docker!) send window drawing requests to it.
      \end{itemize}
    \end{frame}

    % https://tex.stackexchange.com/questions/156480/updating-picture-after-pause
    \begin{frame}
      \frametitle{Baseline}
      \begin{columns}[T]
        \begin{column}{0.5\textwidth}
          \begin{overprint}
          \includegraphics<1| handout:0>[width=\textwidth,height=0.85\textheight,keepaspectratio]{../graphics/020.eps}
          \includegraphics<2| handout:0>[width=\textwidth,height=0.85\textheight,keepaspectratio]{../graphics/030.eps}
          \includegraphics<3| handout:0>[width=\textwidth,height=0.85\textheight,keepaspectratio]{../graphics/040.eps}
          \includegraphics<4| handout:0>[width=\textwidth,height=0.85\textheight,keepaspectratio]{../graphics/050.eps}
          \includegraphics<5->[width=\textwidth,height=0.85\textheight,keepaspectratio]{../graphics/060.eps}
          \end{overprint}
        \end{column}
        \begin{column}{0.5\textwidth}
          \begin{overprint}
          \begin{itemize}
          \item<1-> Windows
          \item<1-> X Server
          \item<2-> WSL
          \item<3-> User and Home Directory
          \item<4-> Docker CLI
          \item<5-> Docker
          \end{itemize}
          \end{overprint}
        \end{column}
      \end{columns}
      \note<2->[item]{Whenever I say ``Host'' (unqualified), I'm referring to WSL.}
      \note<3->[item]{WSL user, but I'm guessing it maps to a Windows user.}
      \note<3->[item]{WSL directory, but I'm guessing it maps to a Windows directory.}
    \end{frame}

    \begin{frame}
      \frametitle{A Word on Operating Systems}
      \begin{itemize}
      \item<2-> Linux: clearly the best choice for this.
      \item<3-> Windows: surprisingly great.
        \begin{itemize}
        \item<4-> WSL is \textit{very} nice.
        \item<5-> WSL appears to share all resources (CPU; Memory) with Windows.
        \item<6-> VcXsrv is \textit{very} capable.
        \end{itemize}
      \item<7-> OS X: surprisingly bad.
        \begin{itemize}
        \item<8-> Docker desktop ``takes'' resources from host.
        \item<9-> XQuartz was not as nice as VcXsrv --- at least in Ventura.
        \end{itemize}
      \end{itemize}
      \note<2->[item]{I started all this madness on Linux. I ran out of cores.}
      \note<3->[item]{Why did I end up using Windows? Because like most gamers, my Windows PC is the most powerful machine I have.}
    \end{frame}

    \begin{frame}
      \frametitle{Sandbox}
      \begin{columns}[T]
        \begin{column}{0.5\textwidth}
          \begin{overprint}
          \includegraphics<1| handout:0>[width=\textwidth,height=0.85\textheight,keepaspectratio]{../graphics/070.eps}
          \includegraphics<2| handout:0>[width=\textwidth,height=0.85\textheight,keepaspectratio]{../graphics/080.eps}
          \includegraphics<3| handout:0>[width=\textwidth,height=0.85\textheight,keepaspectratio]{../graphics/090.eps}
          \includegraphics<4| handout:0>[width=\textwidth,height=0.85\textheight,keepaspectratio]{../graphics/100.eps}
          \includegraphics<5>[width=\textwidth,height=0.85\textheight,keepaspectratio]{../graphics/110.eps}
          \end{overprint}
        \end{column}
        \begin{column}{0.5\textwidth}
          \begin{overprint}
          \begin{itemize}
          \item<1-> Container
          \item<2-> Programs
          \item<3-> User and Home Directory ...
          \item<4-> ... Mapped to those on the host.
          \item<5-> Docker CLI talks to host's dockerd.
          \end{itemize}
          \end{overprint}
        \end{column}
        \note<2->[item]{Of particular interest:\\Docker CLI\\supervisord}
      \end{columns}
    \end{frame}

    \begin{frame}
      \frametitle{Docker-Beside-Docker}
      \only<1>{If ``Docker-In-Docker'' means nothing to you, tune out for this slide and the next.}
      \begin{itemize}
      \item<2->If, from inside a container, you want to be able to:
        \begin{itemize}
        \item<3->Create new containers
        \item<4->Share files in your container with these new containers
        \end{itemize}
        \item<5->You don't need Docker-in-Docker.
      \end{itemize}
    \end{frame}

    \begin{frame}
      \frametitle{Docker-Beside-Docker}
      \begin{itemize}
      \item<1->Install the docker CLI in your container.
      \item<2->Mount the host's docker unix domain socket into the container.
      \item<3->Ensure that for host directories bind-mounted into your container, the
        container and the host use the same name for them.\\
        e.g.: -v /home/emacdona/projects:/home/emacdona/projects
      \end{itemize}
      \note<3->[item]{If you follow these instructions, containers you create with
        the docker cli in your container --- with bind mounts to directories in
        your container --- will actually be created by the same dockerd your
        container was created with.\\Their bind mounts will technically see files
        on the host, but those will be the same files -- with the same names -- as
        those in your container.}
      \note<3->[item]{Just write a startup script to manage all this for you!
        see: my startup script}
    \end{frame}

    \begin{frame}
      \frametitle{Demo II}
      \begin{center}
        {\Huge Demo II}
      \end{center}
      \note[item]{alias sc\\ sc status}
      \note[item]{'lazydocker'; show DISPLAY env var; show 'top'}
      \note[item]{sc start emacs}
      \note[item]{sc start firefox --- Will be able to enter Intellij auth; Useful for Demo II}
      \note[item]{sc start intellij --- will be able to show ansible config in Demo II}
      \note[item]{sc start qtcreator}
      \note[item]{docker-beside-docker:\\
        cd /home/emacdona/projects/onlispBookClub \\\&\& docker run -it -v `pwd`:`pwd` -w `pwd`
        ubuntu sh -c 'ls | sort'
      }
    \end{frame}

    \begin{frame}
      \frametitle{Kubernetes in the Sandbox}
      \begin{columns}[T]
        \begin{column}{0.5\textwidth}
          \begin{overprint}
          \includegraphics<1| handout:0>[width=\textwidth,height=0.85\textheight,keepaspectratio]{../graphics/120.eps}
          \includegraphics<2>[width=\textwidth,height=0.85\textheight,keepaspectratio]{../graphics/130.eps}
          \end{overprint}
        \end{column}
        \begin{column}{0.5\textwidth}
          \begin{overprint}
          \begin{itemize}
          \item<1-> Another docker container...
          \item<2-> ...Running K3d.
          \end{itemize}
          \end{overprint}
        \end{column}
      \end{columns}
    \end{frame}

    \begin{frame}
      \frametitle{Demo III}
      \begin{center}
        {\Huge Demo III}
      \end{center}
      \note[item]{provision k3d cluster}
      \note[item]{watch lazydocker}
      \note[item]{watch k9s}
      \note[item]{show deployed apps --- if they come up in time}
      \note[item]{deploy lisp ``hello world'' app and show result}
      \note[item]{show ``resource monitor'' to give indication of resources required}
      \note[item]{helm uninstall hw}
      \note[item]{
        init-istio.sh\\
        yes  xargs -I{} sh -c 'curl -s http://helloworld.lisp.test/hi /dev/null'\\
        show isto dashboard (default namespace)}
      \note[item]{reinstall hw (will now have istio sidecar)}
    \end{frame}

    \begin{frame}
      \frametitle{Where to Next?}
      \begin{itemize}
        \item<2->Replace supervisord with systemd --- which may necessitate
          moving to podman.\footnote<2->{\href{https://developers.redhat.com/blog/2019/04/24/how-to-run-systemd-in-a-container}{https://developers.redhat.com/blog/2019/04/24/how-to-run-systemd-in-a-container}}
        \item<3->Self host in the Gitlab instance that's deployed in the sandbox.
        \item<4->LDAP server to mimic Active Directory --- and serve as
          authoritative identity source for Keycloak...
        \item<5->... SSO across the cluster.
        \item<6->Terraform to provision AKS\footnote<6->{Azure Kubernetes Service},
          EKS\footnote<6->{Elastic Kubernetes Service (Amazon)}, GKE\footnote<6->{Google
            Kubernetes Engine} deployment target environments.
        \item<7->More cores, more memory!
      \end{itemize}
      \note<2->[item]{Nothing wrong with supervisord, but systemd is a much
        more complete solution for managing process lifecycles.}
      \note<7->[item]{I understand now how one can spend \$10k on a workstation}
    \end{frame}

    \begin{frame}
      \frametitle{Something to Think About}
      \begin{itemize}
      \item<2->We used to ship our application code (jars, wars, ears, etc).
      \item<3->Then we shipped containers --- application code plus runtimes and dependencies.
      \item<4->What's preventing us from shipping complete infra (CI/CD,
        integrated identity provider) and individual developer environments with our software?
      \end{itemize}
    \end{frame}

    \begin{frame}
      \frametitle{Resources and Tools Used}
      \begin{itemize}
      \item Presentation
        \begin{itemize}
        \item \LaTeX:
          \href{https://www.latex-project.org/}{https://www.latex-project.org/}
        \item Beamer:
          \href{https://ctan.org/pkg/beamer?lang=en}{https://ctan.org/pkg/beamer?lang=en}
        \item Pympress:
          \href{https://github.com/Cimbali/pympress}{https://github.com/Cimbali/pympress}
        \item 12-bit Rainbow Palette:
          \href{https://iamkate.com/data/12-bit-rainbow/}{https://iamkate.com/data/12-bit-rainbow/}
        \end{itemize}
      \item Everything else
        \begin{itemize}
        \item VcXsrv:
          \href{https://sourceforge.net/projects/vcxsrv/}{https://sourceforge.net/projects/vcxsrv/}
        \item supervisord:
          \href{http://supervisord.org/}{http://supervisord.org/}
        \item k3d:
          \href{https://k3d.io/}{https://k3d.io/}
        \item lazydocker:
          \href{https://github.com/jesseduffield/lazydocker}{https://github.com/jesseduffield/lazydocker}
        \item k9s:
          \href{https://k9scli.io/}{https://k9scli.io/}
        \item spacemacs:
          \href{https://github.com/syl20bnr/spacemacs}{https://github.com/syl20bnr/spacemacs}
        \end{itemize}
      \end{itemize}
    \end{frame}

    \begin{frame}
      \frametitle{Questions}
      \begin{center}
        {\Huge Questions ?}
      \end{center}
    \end{frame}

\end{document}
